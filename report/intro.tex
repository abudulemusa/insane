\chapter{Introduction} \label{chap:intro}
%Pointer analysis is a static analysis technique that builds information on the
%relations between pointers and allocated objects. It is also often referred to
%as points-to or alias analysis.
%
%In most object oriented languages such as Scala, the use of pointers is
%overwhelming, rendering even basic static analyses techniques brittle. It is
%thus often necessary to establish information on the aliasing relations between
%variables, as well as some knowledge of the shape of structures stored in the
%heap.

Abstract Interpretation \cite{DBLP:conf/popl/CousotC77,DBLP:conf/popl/CousotC02}.

\section{Contributions}

\section{Organization of this thesis}
The rest of this thesis is organized as follows: Then,
Chapter~\ref{chap:overview} gives a quick overview of the tool, followed by
in-depth description of the initial analysis phases.
Chapter~\ref{chap:pointer} describes in full details the pointer and effect
analysis phase.  In Chapter~\ref{chap:implementation}, we describe some
technical implementation details. Chapter~\ref{chap:related} describes previous
work done in the field of pointer and effect analysis. We then conclude in
Chapter~\ref{chap:conclusion} with some ideas for future work.
