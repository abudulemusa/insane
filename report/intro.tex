\chapter{Introduction} \label{chap:intro}
Pointer analysis is a static analysis technique that builds information on the
relations between pointers and allocated objects. It is also often referred to
as points-to or alias analysis.
In object-oriented languages such as Scala, the use of pointers is
pervasive, rendering even basic static analyses techniques brittle. It is
thus often necessary to establish information on the aliasing relations between
variables, as well as some knowledge of the shape of structures stored on the
heap. This then enables opportunities to run more analyses or to perform
compiler optimizations.

Effect analyses attempt to summarize the side effects of procedures in a certain
domain. In this thesis, we focus on memory-based effects, and are thus interested
in computing a summary of read and write operations performed on object fields.
Clearly, any such effect analysis needs to rely on a good pointer analysis,
and vice-versa. For this reason, we perform both analyses side by side.

The summary of effects coupled with aliasing information can later be used to
perform various kinds of optimizations or enable more sophisticated analyses.
For instance, if we establish that two sequential operations affect disjoint
parts of the heap, we could safely run them in parallel. Also, given a precise
alias information, we could perform some form of typestate analysis, which
consists in checking that objects are used following a certain protocol. A
typical example is objects representing files: it is required that you first
open a file before reading from it. Such
analyses\cite{DBLP:journals/tosem/FinkYDRG08} require a precise alias analysis
to limit the amount of spurious warnings.

%improve
Our analysis is based on abstract interpretation
\cite{DBLP:conf/popl/CousotC77,DBLP:conf/popl/CousotC02}. The abstract
representation consists of graphs and is closely based on \cite{Salcianu2006}.
Such graphs are built so that the analysis is compositional.

\section{Contributions}
This thesis makes the following contributions:
\begin{itemize}
    \item
    We present an inter-procedural effect and alias analysis for the Scala
programing language. Our analysis works on arbitrary Scala code, assuming the
absence of concurrency and provided that the complete source code is available.
Our analysis builds on previous work on inter-procedural pointer analysis for
Java \cite{Salcianu2006}. 
We adapted and extended the precision and scope of the original technique with the following features:
    \begin{itemize}
        \item A differentiation between strong and weak field updates to detect
definitely destructive assignments.
        \item A refinement of the allocation
site abstraction that summarizes sets of object;
by incorporating part of the call-stack information in the labelling of
allocation sites, we increase the precision of the analysis for some common
patterns, such as factory methods.
        \item A recency abstraction to be able to determine when an allocation
site abstracts a unique object (singletons).
    \end{itemize}

    \item We have implemented our analysis in a tool called {\insane} whose
source code is freely
available.\footnote{\url{http://github.com/colder/insane}} {\insane} extends
the official Scala compiler and can thus in principle be used with any Scala
program.
% Notable features include:
%     \begin{itemize}
%         \item A backend storage system for intermediate graph results using a
%         database.
%         \item A simple way to describe the effects of unanalyzable methods, such
%         as the ones from the Java library.
%     \end{itemize}
\end{itemize}

\section{Outline}
The rest of this thesis is organized as follows: Chapter~\ref{chap:overview}
gives a quick overview of the tool, followed by an in-depth description of the
initial analysis phases. Chapter~\ref{chap:pointer} describes in full details
the pointer and effect analysis phase. In Chapter~\ref{chap:implementation},
we describe implementation details. Chapter~\ref{chap:related} describes
previous work done in the field of pointer and effect analysis. We then
conclude in Chapter~\ref{chap:conclusion} with some ideas for future work.
