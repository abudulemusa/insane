% -*-latex-*-
% $Log: cover.tex,v $
% Revision 1.7  2001/02/08 18:53:16  boojum
% changed some \newpages to \cleardoublepages
%
% Revision 1.6  1999/10/21 14:49:31  boojum
% changed comment referring to documentstyle
%
% Revision 1.5  1999/10/21 14:39:04  boojum
% *** empty log message ***
%
% Revision 1.4  1997/04/18  17:54:10  othomas
% added page numbers on abstract and cover, and made 1 abstract
% page the default rather than 2.  (anne hunter tells me this
% is the new institute standard.)
%
% Revision 1.4  1997/04/18  17:54:10  othomas
% added page numbers on abstract and cover, and made 1 abstract
% page the default rather than 2.  (anne hunter tells me this
% is the new institute standard.)
%
% Revision 1.3  93/05/17  17:06:29  starflt
% Added acknowledgments section (suggested by tompalka)
%
% Revision 1.2  92/04/22  13:13:13  epeisach
% Fixes for 1991 course 6 requirements
% Phrase "and to grant others the right to do so" has been added to
% permission clause
% Second copy of abstract is not counted as separate pages so numbering works
% out
%
% Revision 1.1  92/04/22  13:08:20  epeisach
\title{Toward Interprocedural Effect and Pointer Analysis for Scala}

\author{Etienne Kneuss}

\prevdegrees{
BSc., Computer Science\\
\'{E}cole Polytechnique F\'{e}d\'{e}rale de Lausanne (2009) }

\department{School of Computer and Communication Sciences}
% If the thesis is for two degrees simultaneously, list them both
% separated by \and like this:
% \degree{Doctor of Philosophy \and Master of Science}
\degree{Master of Science in Computer Science}
\degreemonth{August}
\degreeyear{2011}
\thesisdate{June 24, 2011}

%% By default, the thesis will be copyrighted to MIT.  If you need to copyright
%% the thesis to yourself, just specify the `vi' documentclass option.  If for
%% some reason you want to exactly specify the copyright notice text, you can
%% use the \copyrightnoticetext command.
%\copyrightnoticetext{\copyright IBM, 1990.  Do not open till Xmas.}

% If there is more than one supervisor, use the \supervisor command
% once for each.
\supervisor{Viktor Kuncak}{Professor}
\supervisor{Philippe Suter}{Ph.D. Student}

% This is the department committee chairman, not the thesis committee
% chairman.  You should replace this with your Department's Committee
% Chairman.
\chairman{Prof. XXX}{Associate Department Head\\Chair,
Committee on Graduate Students}

% Make the titlepage based on the above information.  If you need
% something special and can't use the standard form, you can specify
% the exact text of the titlepage yourself.  Put it in a titlepage
% environment and leave blank lines where you want vertical space.
% The spaces will be adjusted to fill the entire page.  The dotted
% lines for the signatures are made with the \signature command.
\maketitle

% The abstractpage environment sets up everything on the page except
% the text itself.  The title and other header material are put at the
% top of the page, and the supervisors are listed at the bottom.  A
% new page is begun both before and after.  Of course, an abstract may
% be more than one page itself.  If you need more control over the
% format of the page, you can use the abstract environment, which puts
% the word "Abstract" at the beginning and single spaces its text.

%% You can either \input (*not* \include) your abstract file, or you can put
%% the text of the abstract directly between the \begin{abstractpage} and
%% \end{abstractpage} commands.

% First copy: start a new page, and save the page number.
\cleardoublepage
% Uncomment the next line if you do NOT want a page number on your
% abstract and acknowledgments pages.
% \pagestyle{empty}
\setcounter{savepage}{\thepage}
\begin{abstractpage}
Static program analysis techniques working on object-oriented languages usually
require precise knowledge of the aliasing relation between variables. This
knowledge is, amongs other things, important to understand the actual effects
of method calls on objects. We developed an analysis combining a pointer analysis
coupled with a memory-based effects analysis for the Scala programming
language. This analysis is based on abstract interpretation, and represents the
effects of methods in a graph representation. This representation allows the
analysis to be compositional. We also provide the implementation of this
analysis in \insane, a freely-available extension of the Scala compiler.

\end{abstractpage}

% Additional copy: start a new page, and reset the page number.  This way,
% the second copy of the abstract is not counted as separate pages.
% Uncomment the next 6 lines if you need two copies of the abstract
% page.
% \setcounter{page}{\thesavepage}
% \begin{abstractpage}
% Static program analysis techniques working on object-oriented languages usually
require precise knowledge of the aliasing relation between variables. This
knowledge is, amongs other things, important to understand the actual effects
of method calls on objects. We developed an analysis combining a pointer analysis
coupled with a memory-based effects analysis for the Scala programming
language. This analysis is based on abstract interpretation, and represents the
effects of methods in a graph representation. This representation allows the
analysis to be compositional. We also provide the implementation of this
analysis in \insane, a freely-available extension of the Scala compiler.

% \end{abstractpage}

\cleardoublepage

\section*{Acknowledgments}
First and foremost I would like to thank my supervisor, Viktor Kuncak, for his
sustained enthousiasm, inspired suggestions and exemplary guidance thourough
the course of this thesis. I would also like to thank him for giving me the
opportunity to work on such interesting subjects, and look forward to working
with him as I continue with my research. I would also like to thank Philippe
Suter for his continuous feedback and guidance without which this thesis would
not have been possible.

I would like to extend my sincere thanks to the members of the Laboratory for
Automated Research and Reasonning for numerous stimulating discussions: Ali,
Andrej, Eva, Giuliano, Hossein, Ruzica, Swen and Tihomir: thanks. I would also
like to thank Lukas Rytz for his patience and technical support regarding the
Scala compiler. Finally I thank those who continuously supported me; my
family, my friends, and of course Aline.




%%%%%%%%%%%%%%%%%%%%%%%%%%%%%%%%%%%%%%%%%%%%%%%%%%%%%%%%%%%%%%%%%%%%%%
% -*-latex-*-
