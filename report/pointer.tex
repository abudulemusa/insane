\chapter{Pointer and Effect Analysis}
\label{chap:pointer}

\section{Graph Semantics}

\section{Graph-based Type Analysis}
Even though our global type analysis was flow sensitive, it suffered from major
imprecision with respect to fields and arguments. For this reason, we
implemented a type analysis based on our graphs. The approach we took is to
attach type information to each node. Since nodes represent objects, this is a
sensible thing to do. The additional information we store alongside each node
is similar to what we used in our previous type analysis: $(T_{sub}, T_{exa})$
a pair of two sets of types $T_{sub}$ and $T_{exa}$ where $T_{sub}$ represents
the set of types from which we need to include subtypes, and $T_{exa}$.
The set of runtime types attached to each node is determined based on the node
types. Figure~\ref{fig:pt:types} illustrates the main cases. We then use those
types in order to compute the set of potential targets for a method call. Given
the call \verb/obj.foo()/, we obtain the set of runtime types corresponding to
\verb/obj/ as follows:
$$
    types(\verb/obj/) = \bigcup \{ types(n) ~|~ n \in nodes(\verb/obj/) \}
$$
we can then look for potential targets in the resulting set of types, like we
did in our previous type analysis.


\begin{figure}[h]
    \centering

    \begin{tabular}{ l | l }
        Node Type & Types Associated \\
        \hline
        INode(A)           & $\{\{\}, \{A\}\}$ \\
        LNode(a.f)         & $\{\{type(a.f)\}, \{type(a.f)\}\}$ \\
        PNode(arg)         & $\{\{type(arg)\}, \{type(arg)\}\}$ \\
        OBNode(A)          & $\{\{\},   \{A\}\}$ \\
        NNode              & $\{\{\},   \{\}\}$ \\
        GBNode             & $\{\{Object\},   \{Object\}\}$ (all)\\
        Literal Nodes      & $\{\{\},   \{type(Literal)\}\}$\\
    \end{tabular}

    \caption{Types associated to nodes}
    \label{fig:pt:types}
\end{figure}

If we consider that each field read yield a \emph{load node}, and that every
arguments become \emph{parameter nodes}, this type analysis is exactly as
precise as the one described previously. The main difference comes when those
graphs get inlined. Indeed, during inlining, parameter nodes are mapped to
other nodes at the call site, and load nodes are resolved. The natural inlining
of methods makes this type analysis context sensitive. We consider in
Figure~\ref{fig:pt:precise} an example illustrating this improvement in precision.

\begin{figure}[h]
    \centering
\begin{lstlisting}
class A {
  var f: A = null
  def setF(a: A) { f = a }

  def test(obj: A) {
    val a = new A
    a.setF(a)
    a.f.foo()
  }
  def foo() {
    println("A")
  }
}

class B extends A {
  override def foo() {
    println("B")
  }
}
\end{lstlisting}
    \caption{Improved type analysis}
    \label{fig:pt:precise}
\end{figure}

At the time of the call to \verb/a.f.foo()/ we have from the graph at that
program point that \verb/a.f/ is the \emph{inside node} corresponding to the
object from \verb/new A/. We thus obtain $\{ \{\}, \{A\} \}$ for \verb/a.f/ instead of
$\{ \{A\}, \{A\} \}$, which excludes $B.foo$ from the call.
