\chapter{Related Work}
\label{chap:related}
To our knowledge, we are the first trying to perform such alias analysis for
Scala. Although Scala compiles to Java bytecode, and thus any analyzer working
with bytecode could be used for analyzing Scala, the steps performed during
compilation introduce many artifacts. For that reason, an analysis focued on
Scala will be able to provide much more useful and precise result than one
working with arbitrary Java bytecode.

Given the usefulness of alias analysis, has been constantly worked on in the
past decades and remains an active area. Most of the time, alias analysis is not
the goal but the mean to achieve a more sophisticated analysis.

The work that is naturally the most related to this thesis is the work done by
Alexandru Salcianu in \cite{Salcianu2001,Salcianu2006}, on which this thesis is
closely based. They provide a compositional graph based pointer analysis that
focus on establishing escaping information. While we also provide a similar
compositional analysis based on graphs, we assign slightly different semantics
to our graphs to cope with strong updates, that they did not support.  We also
handle program points refinement, which allows us to provide a more precise
analysis in the presence of factory methods.

In \cite{DBLP:conf/oopsla/DilligDA10}, they propose to add invariants to
refine pointer relations in the heap. Our analysis is thus less precise in
that regard, as it is currently not path sensitive at all. In
\cite{DBLP:conf/ecoop/ChalinJ07}, they develop a demand-driven alias analysis.
It is however flow insensitive and thus less precise that what we have here.
However, the fact that it is demand-driven is very interesting.

In
\cite{DBLP:conf/oopsla/DiwanMM96,DBLP:conf/oopsla/BaconS96,DBLP:conf/fossacs/JensenS01},
they describe similar type analysis techniques to compute the call graph in the
presence of dynamic dispatch.

One distinction between our work and \cite{Salcianu2006} is that we
differentiate between weak and strong updates. Researchers have looks at ways
to further refine this paradigm by introducing logical predicates that qualify
the uniqueness of the object summary \cite{DBLP:conf/esop/DilligDA10}; although
the technique appears appealing, we do not expect that it would provide much
benefit in our setting. Indeed, our analysis already treats most updates as
strong, and it is not clear whether parametrizing them with predicates would
allow us to perform a strong update when we previously could not.

Different alias analysis techniques have been explored in order to perform type
state analysis \cite{DBLP:journals/tse/StromY86, DBLP:journals/tse/StromY93}.
In \cite{DBLP:conf/pldi/FahndrichD02}, instead of figuring out heap aliasing,
required for type state checking, they propose a type system extension that
inherently restricts aliasing interferences. In
\cite{DBLP:conf/ecoop/HallerO10}, they design an annotation system to describe
messages passed between concurrent objects so that they can be used
without any risk of race-conditions.

Much work has been done in order to obtain guarantees during object
initialization
\cite{DBLP:conf/popl/QiM09,DBLP:conf/oopsla/FahndrichX07,DBLP:conf/ecoop/ChalinJ07}.
For instance, \cite{DBLP:conf/oopsla/FahndrichX07} proposes type-based
techniques to prevent issues from arising during the initialization phase. In
our analysis, we handle initialization in a straightforward manner by inlining
the graphs from the various constructors. This allows us to detect whether at
some point during or after initialization, an object field retain its default
value (i.e.  null).

Constructing a precise call-graph in the presence of higher order functions has
been known to be a problematic analysis. It has been shown that it is in fact
equivalent to dynamic dispatch analysis \cite{DBLP:conf/pldi/MightSH10}. In
Scala, the link between the two is explicit, since closures get compiled into
classes defining an apply method. We could thus benefit from ideas developed by
\cite{DBLP:conf/pldi/Shivers88} and later refined by others to improve our
dynamic dispatch analysis for calls to closures.
