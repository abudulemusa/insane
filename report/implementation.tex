\section{Implementation}
This tool has been implemented on the top of the Scala compiler, and can be
built as a plugin of the compiler. One of the main advantage of building it as
a compiler plugin is that it grants us immediate access to the trees and all
type information that we need. The compiler allows us to plug our tool between
two existing phases. Depending on where the plugin is inserted, it dramatically
changes the aspect of the trees. In this case, we decided to put it late in the
compilation process, so that we would not have to deal with closures, inner
classes, mixins, or genericity. This however comes with some cost: we lost
precision in the presence of parametric types, and the amount of code to
analyze is much bigger.

\subsection{Class Hierarchy}
The first implementation problem we faced while working with the Scala compiler
is that, it does not provide any way to access subtypes of one symbol, only its
super type. For this reason, we had to traverse every symbols defined in the
classpath in order to reconstruct the entire hierarchy, which allowed us to
compute subtypes. This process of traversing every defined symbols in the class
path is costly (minimum 30'000 symbols given that the Java and Scala library are always
included), and represented most of the analysis time for small examples.

For this reason, we decided to store the class hierarchy of the Scala and Java
libraries into a database. We use a nested
set\footnote{\url{http://dev.mysql.com/tech-resources/articles/hierarchical-data.html}}
representation for our hierarchical data, which allows us to retrieve the entire
set of subtypes in one SQL query efficiently.
