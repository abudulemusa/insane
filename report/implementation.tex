\chapter{Implementation}
\label{chap:implementation}
This tool has been implemented on the top of the Scala compiler, and can be
built as a plugin of the compiler. One of the main advantage of building it as
a compiler plugin is that it grants us immediate access to the trees and all
type information that we need. The compiler allows us to plug our tool between
two existing phases. Depending on where the plugin is inserted, it dramatically
changes the aspect of the trees. In this case, we decided to put it late in the
compilation process, so that we would not have to deal with closures, inner
classes, mixins, or genericity. This however comes with some cost: we lost
precision in the presence of parametric types, and the amount of code to
analyze is much bigger.

\section{Class Hierarchy}
The first implementation problem we faced while working with the Scala compiler
is that, it does not provide any way to access subtypes of one symbol, only its
super type. For this reason, we had to traverse every symbols defined in the
classpath in order to reconstruct the entire hierarchy, which allowed us to
compute subtypes. This process of traversing every defined symbols in the class
path is costly (minimum 30'000 symbols given that the Java and Scala library are always
included), and represented most of the analysis time for small examples.

For this reason, we decided to store the class hierarchy of the Scala and Java
libraries into a database. We use a nested
set\footnote{\url{http://dev.mysql.com/tech-resources/articles/hierarchical-data.html}}
representation for our hierarchical data, which allows us to retrieve the entire
set of subtypes in one SQL query efficiently.

\section{Pointer and Effect Analysis}
As we've seen in Chapter~\ref{chap:pointer}, our analysis computes points-to
graphs for each method present in the analyzed source. When calling a method,
we inline the graph corresponding to the target method into the caller graph.

\subsection{Library Dependencies}
One of the major problems that we faced while trying to analyze even small
Scala programs is that references to the Scala library are ubiquitous even if
they are not explicitely stated in the source. This is caused by the various
compiler phases prior to ours, responsible for desugaring Scala code. To
illustrate this issue, we consider an apparently self-contained example in
Figure~\ref{fig:imp:class}, and the corresponding code at the time of our phase
in Figure~\ref{fig:imp:classlater}. We can see that even though the original
code made no explicit references to the Scala library, the actual code that we
analyze does.
\begin{figure}[h]
    \centering
\begin{lstlisting}
class A (val next: A*)

object Test {
  def test() = {
    val a = new A()
    val b = new A(a)
  }
}
\end{lstlisting}
    \caption{Self-contained example in Scala}
    \label{fig:imp:class}
\end{figure}

\begin{figure}[h]
    \centering
\begin{lstlisting}
package <empty> {
  class A extends java.lang.Object with ScalaObject {
    private[this] val next: Seq = _
    def next(): Seq = A.this.next
    def this(next: Seq): A = {
      A.this.next = next
      A.super.this()
      ()
    }
  }
  final object Test extends java.lang.Object with ScalaObject {
    def test(): Unit = {
      val a: A = new A(immutable.this.Nil)
      val b: A = new A(scala.this.Predef.wrapRefArray(
            Array[A]{a}.$asInstanceOf[Array[java.lang.Object]]()))
      ()
    }
    def this(): object Test = {
      Test.super.this()
      ()
    }
  }
}
\end{lstlisting}
    \caption{Corresponding code ar our compiler phase}
    \label{fig:imp:classlater}
\end{figure}

\subsection{Storing Intermediate Results}
Those ubiquitous references to the Scala library forces us to analyze the
library along with the code. Given that the library consists of approximately
90'000 methods, it would be very inefficient to require the library code to be
included alongside each piece of code we want to analyze.

Thankfully, the modularity of the graph-based effects representation allows us
to pre-calculate the graphs for every methods of the library and store them in
a database. To store those results in the database, we had to implement a
special serialization procedure, as the state stored in the graphs contained
references to internal compiler structures that were not serializable. Our custom
serialization allows for full recovery of the state.

It is worth noting that this pre-calculation is only possible under the
assumption that the effects of library methods are "self-contained". This is better
expressed in terms of the shape of the call-graph: there should be no calls
from the library to non-library methods. This assumption generally does not
hold, in the presence of higher-order functions, or callbacks. This is covered in more
details in Section~\ref{sec:con:limitations}.

\subsection{Unanalyzable Methods}
While analyzing the library, we stumbled upon the fact that it is itself not
self-contained but heavily references the Java library. In overall, it calls
over 700 distinct methods from the Java library.

Even though we could in theory apply the same analysis to the Java source code, our
analysis was implemented on top of the Scala compiler, which is unable to
compile Java source code. We thus were not able to apply the same
pre-calculation technique used for the Scala library directly.

Instead of making conservative assumptions and applying \emph{havoc} on every
objects involved in those various library calls, we provided a way for the user
to provide Scala implementations of those Java classes and methods. This is
done via annotations put at the level of the classes and methods, that will
inform our compiler that this graph is meant to represent a different methods.
We illustrate this use of annotations in Figure~\ref{fig:imp:annotations} with
an excerpt of the \verb=java.math.BigInteger= Scala implementation.
\begin{figure}[h]
    \centering
\begin{lstlisting}
package insane
package predefined

import annotations._

@AbstractsClass("java.math.BigInteger")
class javamathBigInteger {
  @AbstractsMethod("java.math.BigInteger.abs(()java.math.BigInteger)")
  def abs(): java.math.BigInteger = {
    new java.math.BigInteger("42")
  }

  // ...

  @AbstractsMethod("java.math.BigInteger.valueOf"+
                "((x$1: Long)java.math.BigInteger)")
  def valueOf(x1: Long): java.math.BigInteger = {
    new java.math.BigInteger(x1)
  }
}
\end{lstlisting}
    \caption{Dummy Scala implementation of java.math.BigInteger using annotations}
    \label{fig:imp:annotations}
\end{figure}
